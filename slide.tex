\documentclass[dvipdfmx,10pt]{beamer}
\usepackage{amsmath,amssymb}
\usepackage{bbm}
\usepackage{graphicx}
\usepackage{url}
% \usepackage{amsthm}
\newcommand{\bo}[1]{{\boldsymbol #1}}
\newcommand{\dx}{\,{\rm d}}
\newcommand{\difall}[2]{\frac{{\rm d}{#1}}{{\rm d}{#2}}}
\newcommand{\parpar}[2]{\frac{\partial{#1}}{\partial{#2}}}
\newcommand{\pmat}[1]{{\begin{pmatrix}#1\end{pmatrix}}}
\renewcommand{\div}{\mathop{\rm div}\nolimits}
\usetheme{Copenhagen}
\title{平衡探索木の入替え操作における\\B+木の高速性}
\author{sa8}
\begin{document}
\maketitle
% TODO: Pietのroll操作について言及
% TODO: 2分木で発生するキャッシュミスの恐れ
\section{平衡探索木とは?}
\begin{frame}
    \frametitle{2分探索木}
    \begin{block}{2分木\cite[\S 6.3]{kondo:1998}}
        \begin{enumerate}
            \item 空の木は二分木である。
            \item 次のいずれかを満足する節のみからなる木は、二分木である:
                  \begin{itemize}
                      \item 子をもたない
                      \item 左の子のみをもつ
                      \item 右の子のみをもつ
                      \item 左右2つの子をもつ
                  \end{itemize}
        \end{enumerate}
    \end{block}
    各節の左の子を根とする部分木を{\bf 左部分木}と呼ぶ。{\bf 右部分木}も同様。
    \begin{block}{2分探索木\cite[\S 9]{kondo:1998}}
        各節に持たせたデータが次の関係を満たす二分木:

        任意の節$\tt x$について、左部分木の要素は節$\tt x$よりも小さく、
        右部分木に含まれる要素は節$\tt x$よりも大きい。
    \end{block}
\end{frame}
\begin{frame}
    % TODO: リスト構造と同型になった2分木の図。
    \frametitle{平衡2分探索木}
    \begin{block}{平衡2分探索木}
        木の高さの偏りを抑える構造を持つ2分探索木。
    \end{block}
    \begin{exampleblock}{例}
        AVL木、Splay木、赤黒木
    \end{exampleblock}

\end{frame}
\bibliographystyle{jplain}
\bibliography{biblio}
\end{document}